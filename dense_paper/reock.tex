\section{Reock}\label{sec:reock}
\subsection{The Stereographic Projection}
Before analyzing Reock scores, we preform a preliminary 
discussion about the stereographic projection.
\begin{definition}
  The \textbf{stereograpic projection} from the north pole $N$ is
  defined by placing a copy $\R^2$ tangent to the sphere at the south
  pole, and sending any $p\in \mbb{S}^2\ssm \{N\}$ to 
  the unique point $q$ in $\R^2$ such that $q\in \overline{Np}$.
\end{definition}
\begin{remark}
  It is a classical result that this projection sends every cap on
  the sphere which doesn't pass through the north pole to a circle in
  the plane. 
\end{remark}
\begin{lemma}
  The stereographic projection sends every cap which does not pass
  through the north pole to a circle in the plane.
\end{lemma}

\begin{lemma}
  Let $f:\Omega\to V$ be a bijection between two planar regions 
  sending generalised circles to generalised circles. 
  Then $f$ is a M\"{o}bius transformation composed with a 
  reflection.
\end{lemma}
This is a slight rewording of a celebrated result by 
Carath\'{e}odory\cite{caratheodory}:
\begin{theorem}
  Every arbitrary bijection of a disc $D$ to a bounded set $D'$ by which
  circles lying completely in $D$ are transformed into circles 
  lying in $D'$ must be of the form $M$ or $\bar M$, where $M$ is a 
  M\"{o}bius transformation.
\end{theorem}
The proof of the lemma then follows:
\begin{proof}
  By Carath\'{e}odory's theorem, we know that on any open 
  ball in $\Omega$, $f=M$ or $f=\bar M$, where 
  $M$ is a M\"{o}bius transformaion. Thus, $f$ or $\bar f$ 
  are holomorphic, so by applying the identity theorem, 
  it follows that $f = M$ or $f = \bar M$ on the entirety of 
  $\Omega$.
\end{proof}

This gives us the following characterization:
\begin{theorem}\label{thm:stereographic_mobius}
  The map projections from the sphere to the plane which send every
  cap to a circle are exactly those which can be written as the
  composition of the stereographic projection followed by a 
  M\"{o}bius transoformation and a reflection.
\end{theorem}
\begin{proof}
  Let $\varphi$ and $\psi$ be two map projections which preserve
  circles, and without loss of generality let $\varphi$ be the
  standard stereographic projection.  Then the composition
  $M^{-1}=\psi\circ\varphi^{-1}$ is a transformation of 
  the plane which preserves circles, so by the previous lemma, 
  $M^{-1}$ is a M\"{o}bius transformation composed with a reflection. 
  Then, we can write $\psi= M\varphi$, which is the
  composition of the stereographic projection and a scaled isometry.
\end{proof}
\subsection{The Reock Compactness Score}
Let $\mathrm{circ}(\Omega)$ denote the \textit{smallest bounding
circle} (smallest bounding \textit{cap} on the sphere) of a region
$\Omega$.  Then the \textit{Reock score} of $\Omega$ is 

$$\mathrm{Reock}(\Omega)=
\frac{\mathrm{area}(\Omega)}{\mathrm{area}(\mathrm{circ}(\Omega))}.$$

Since $\mathrm{Reock}(\Omega)=1$ if and only if $\Omega$ is a circle,
we can observe that any candidate $\vphi$ for a map projection 
which preserves the ordering of Reock scores must, at the 
very least, preserve the maximizeres in the ordering, meaning 
it sends caps on the sphere to circles in the plane.  

By Theorem~\ref{thm:stereographic_mobius}, it follows that 
we can assume that $\vphi=R\circ M \circ \psi$, 
where $R$ is a reflection, $M$ is a M\"{o}bius 
transformation, and $\psi$ is the standard 
stereographic projection. However, we note that reflections do not 
change any compactness scores, so we only need 
to deal with $M\circ \psi$.

We now treat the M\"{o}bius transformation composition 
in the following manner:
\begin{claim}
  There exists a rigid transformation of the 
  sphere, $T_M$, such that $M\psi = T_M\psi$.
\end{claim}
This follows by writing $T_M$

In other words, to show that $M\psi$ 
does not preserve Reock score rankings 
in some region $U$, it suffices to show that 
$\psi$ does not preserve Reock score 
rankings in the transformed region 
$T(U)$.

\begin{theorem}\label{thm:reock}
  Let $A$ be a region on the sphere.  Then there exist two regions
  $\kappa'_N,\kappa'_S\subset A$ such that the Reock scores of
  $\kappa'_N$ and $\kappa'_S$ are equal on the sphere, but under the
  stereographic projection $\varphi$, the Reock score of $\kappa'_S$
  is strictly greater than that of $\kappa'_N$. 
\end{theorem}

\begin{proof}
  First, since $A$ is a region, we can find some cap $\kappa$ inside
  of $A$ such that the center of $\kappa$ is not the south pole of the
  sphere, and the north pole is exterior to $\kappa$.  This cap is
  bisected by a line of longitude, in particular the great circle
  which passes through the two poles and the center of $\kappa$.  This
  line of longitude meets $\kappa$ at exactly two points, and we call
  $p_N$ the point closer to the north pole and $p_S$ the point closer
  to the south pole.

  Choose some $r$ strictly less the radius of $\kappa$ and let
  $\kappa'_S$ be the region constructed by deleting a cap of radius
  $r$ from the interior of $\kappa$ tangent at $p_S$.  Construct
  $\kappa'_N$ analogously by deleting a cap of radius $r$ tangent at
  $p_n$. Observe that the Reock scores of $\kappa'_N$ and $\kappa'_S$
  are identical, since each has the boundary of $\kappa$ as its
  smallest bounding cap and both figures have the same area.

  We now consider the images of $\kappa'_N$ and $\kappa'_S$ under the
  stereographic projection $\varphi$.  Since $\varphi$ preserves
  points of tangency, containment, and sends every cap away from the
  north pole to a circle, the images $\varphi(\kappa'_N)$ and
  $\varphi(\kappa'_S)$ are both regions which are disks with a smaller
  disk, tangent to a point on the circumference, deleted.
  Furthermore, since the boundary of $\kappa$ was the smallest
  bounding cap of both regions on the sphere, the image of the
  boundary of $\kappa$ under $\varphi$ is the smallest bounding circle
  of the images of these regions in the plane.

  We can now observe that these two regions in the plane do not have
  the same Reock score.  Both have the same bounding circle, but
  $\varphi(\kappa'_N)$ is strictly smaller than $\varphi(\kappa'_S)$.
  This is because the stereographic projection distorts areas in a way
  such that figures further from the south pole have their areas
  magnified more than the same region closer to the south pole.  If
  the regions in question are sufficiently small, letting $\theta$
  denote the polar angle at which we consider a small cap on the
  sphere, a cap of radius $r$ will be sent to a circle with radius
  roughly $r/\cos^2{\theta}$\zs{check  this}, and a straightforward
  examination of this as a function of $\theta$ shows that it grows
  faster as $\theta$ increases.

  Since $\varphi(\kappa'_N)$ fills a smaller fraction of the bounding
  circle than $\varphi(\kappa'_N)$ does, its Reock score is strictly
  worse, and the stereographic projection does not preserve Reock
  scores.
\end{proof}
Piecing this together yields:
\begin{theorem}
  There is no map projetion from the sphere to the plane which preserves the ordering of Reock scores.
\end{theorem}
