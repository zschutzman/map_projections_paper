\section{Convex Hull}\label{sec:ch}
A commonly used compactness score is the \textit{convex hull
score}.  We briefly recall the definition of a convex set and then
define this score function.

\begin{definition}
  Let $\mathrm{conv}(\Omega)$ denote the \textit{convex hull} of
  a region $\Omega$ in either the sphere or the plane, which is the
  smallest convex region containing $\Omega$. We define the 
  \textit{convex hull score} of $\Omega$ as 
  \begin{align*}
    \mathrm{CH}(\Omega)=
    \frac{\mathrm{area}(\Omega)}{\mathrm{area}(\mathrm{conv}(\Omega))}
  \end{align*}
\end{definition}
Throughout this section, let $\vphi:\Omega \to \R^2$ be a 
map projection defined on a region $\Omega\subset \mbb{S}^2$.
\begin{lemma}~\label{lem:CH_prep}
  If $\vphi$ preserves $\mc{C}_{CH}$, then the following must 
  hold:
  \begin{enumerate}
    \item $\vphi,\vphi^{-1}$ send convex sets to convex sets
    \item $\vphi$ is a geodesic map.
    \item There exists a region $U\subset \Omega$ 
      such that for any regions $A,B\subset U$, 
      $A$ and $B$ have equal area 
      if and only if $\vphi(A)$ and $\vphi(B)$ have equal 
      area.
  \end{enumerate}
\end{lemma}
\begin{proof}
  \begin{enumerate}
    \item This follows because $\vphi,\vphi^{-1}$ must 
      preserve maximizers of $CH$.
    \item Assume that $\vphi$ sends a geodesic 
      segment $\gamma$ to a nongeodesic segment 
      $\eta$. Then for any $\eps>0$, the 
      $\eps$-neighbourhood of $\gamma$ $N_{\eps}(\gamma)$ 
      is convex, but for sufficiently small $\eps$, 
      $\vphi(N_{\eps}(\gamma)$ is not, a contradiction. 
      Showing that $\vphi^{-1}$ sends geodesic 
      segments to geodesic segments is proved analogously.
      
    \item Let $U\subset \Omega$ be a cap, and 
      let $A,B\subset U$ be regions of equal area. Let 
      $X = U\ssm A$ and $Y = U\ssm B$, and note that 
      $CH(X) = CH(Y)$. Since 
      $\vphi$ preserves $\mc{C}_{CH}$, we 
      must have $CH(\vphi(X)) = CH(\vphi(Y))$. However, 
      since $\vphi(U)$ is convex, it follows that 
      \begin{align*}
        \frac{\mathrm{Area}(\vphi(U))-\mathrm{Area}(\vphi(A))}
        {\mathrm{Area}(\vphi(U))}=
        CH(Y)=CH(X)=
        \frac{\mathrm{Area}(\vphi(U))-\mathrm{Area}(\vphi(B))}
        {\mathrm{Area}(\vphi(U))}
      \end{align*}
      meaning that $\mathrm{Area}(\vphi(A))=\mathrm{Area}(\vphi(B))$. 
      Applying the same argument to $\vphi^{-1}$ 
      proves the result.
  \end{enumerate}
\end{proof}
\begin{theorem}
  There does not exist a map projection satisfying the 
  conditions in Lemma~\ref{lem:CH_prep}
\end{theorem}
\begin{proof}
  Assume that such a map, $\vphi$, exists, and restrict 
  it to $U$ as above. Let $T\subset U$ be a 
  spherical equilateral triangle with 
  $\mathrm{Diam}(T)<\frac{\mathrm{Diam}(U)}{3}$, centered at 
  the center of $U$. Let $T_1$ and $T_2$ be two 
  congruent triangles meeting at a point and 
  each sharing a face with $T$
  \begin{center}
    TODO:picture here
  \end{center}
  \begin{claim}
    The image of $T\cup T_i$ is a parallelogram for any $i$.
  \end{claim}
  \begin{proof}
    Set $i=1$, and note that $T\cup T_1$ is a 
    convex spherical quadrilateral. By symmetry, its geodesic 
    diagonals $D_1,D_2$ on the sphere split it into 
    four equal-area triangles.
    \begin{center}
      TODO: picture here
    \end{center}
    Since $\vphi$ is a geodesic map, it must send 
    $T\cup T_1$ to an Euclidean quadrilateral $Q$ whose diagonals 
    are $\vphi(D_1)$ and $\vphi(D_2)$. Since 
    $\vphi$ sends equal area regions to equal area 
    regions, it follows that the diagonals 
    of $Q$ split it into four equal area triangles, meaning 
    that $Q$ must be a parallelogram.
  \end{proof}
  By the claim, the image of $T\cup T_1\cup T_2$ must 
  consist of two parallelograms, one of whose 
  edges is the diagonal of the other. In other words, 
  $\vphi(T\cup T_1\cup T_2)$ is a quadrilateral trapezoid 
  whose boundary consists of four Euclidean geodesic 
  segments.

  However, since all the angles of $T_i$ and $T$ 
  must be at least $\frac{\pi}{3}$, it follows that 
  the configuration $T\cup T_1\cup T_2$ must consist 
  of at least five geodesic segments, and hence 
  $\vphi$ cannot be a geodesic map.
\end{proof}

