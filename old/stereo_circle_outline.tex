\begin{comment}

\section*{Proof that Stereographic Projection is the Unique-ish Circle-Preserving Projection from the Sphere to the Plane}

\zs{This will all have to be rewritten and integrated above...}

Let $\hat{\C}$ denote the Riemann sphere and $\phi$ be the standard stereographic projection from $\hat{\C}$ to  $\C$ where the north pole ($N$) is mapped to infinity ($\ast$) and the south pole ($S$) to 0.


See \url{http://math.uchicago.edu/~may/REU2013/REUPapers/Kim.pdf} for proofs of Facts 2 and 3.


\begin{fact}
Stereographic projection preserves circles.  Any cap away from $N$ is sent to a circle in the plane, and any cap through $N$ is sent to a line in the plane.  Similarly, any circle or line in the plane is the image of a cap on the sphere under $\phi$.
\end{fact}

\begin{fact}
The group of M\"obius transformations of the extended complex plane, which are generated by the following three actions:
\begin{itemize}
    \item $z\mapsto kz$ (dilation and rotation)
    \item $z\mapsto z+b$ (translation)
    \item $z\mapsto 1/z$ (inversion and reflection)
\end{itemize}
are exactly the set of homeomorphisms of $\hat{\C}$ which preserve caps.  I.e. a homoemorphism of $\hat{\C}$ preserves all caps if and only if it is a M\"obius transformation.  
\end{fact}
\begin{fact}
As a corollary, this is also exactly the set of homeomorphisms of $\C$ which preserve the class of circles and lines, i.e. every circle is sent to either a circle or a line and every line is sent to either a circle or a line.
\end{fact}

\begin{fact}
The set of homeomorphisms of $\C$ which sends every circle to a circle and every line to a line is a strict subset of the M\"obius transformations.
\end{fact}

\zs{Unlike the gnomonic setting, we don't immediately know that all the transformations we are considering have to preserve lines.}
\begin{claim}
There does not exist a homeomorphism of $\C$ which preserves circles but not lines.
\end{claim}
\begin{proof}[Proof Sketch]
Suppose we have a transformation $f$ which preserves circles but not lines, i.e. it is nonlinear.  Then, we can find some line $L$ which is mapped to a curve $f(L)$ which is not a line. We can take three non-collinear points on $f(L)$, call them $f(x),f(y),f(z)$ and consider the unique circle $C$ through those points.  Now, let $f(x')$ be some other point on the circle, and consider the location of $x'$ in the preimage.  If $x'$ is not on L, then we can construct three circles, through ($x',y,z$), ($x',x,z$), and ($x',x,y$), which are all distinct, but whose image must be $C$.  Since any pair of these circles intersects in exactly two points and $f$ must preserve intersections, this is a contradiction.  Therefore, $x'$ must lie on $L$.  Repeating this argument, we see that every point on $C$ must be the image of some point on $L$, but a circle cannot be the homeomorphic image of a line or line segment (by injectivity, we can't be wrapping the line around the circle either), which is our contradiction.  Therefore, preservation of circles implies preservation of lines.
\end{proof}


\begin{fact}
A M\"obius transform of $\C$ can be thought of as using $\phi^{-1}$ to go back to the sphere, moving, rotating, and (maybe) flipping the orientation of the sphere, and reprojecting with $\phi$.
\end{fact}


\begin{theorem}
If $\psi$ is a generalized circle-preserving projection from the sphere to the plane, then $\psi$ is $\phi$ composed with a M\"obius transformation.
\end{theorem}
\begin{proof}

By taking $\phi\circ\psi^{-1}$, we get a generalized circle-preserving transformation of the plane.  By Facts 2, 3, and 4, this is a cap-preserving transformation of $\hat{\C}$ and is therefore a M\"obius transformation, $m$.  Rewriting, we have $m=\phi\circ\psi^{-1}$ so $\psi=m^{-1}\circ \phi$.  Since the M\"obius transformations form a group, $m^{-1}$ is well-defined and also a M\"obius transformation, as desired.

If the composition gives us a map which sends every circle to a circle, then by Claim 1, it also sends every line to a line, so it is a particular kind of M\"obius transformation; a member of the subgroup generated by translation, reflection, scaling, and rotation.


\end{proof}

If we restrict our attention to projections from the lower half-sphere to the plane, we get an analogous statement to the gnomonic one, where the theorem is

\begin{theorem}
Any projection from the lower half-sphere to the plane which sends every cap on the sphere to a circle in the plane is the composition of stereographic projection with a rotation, scaling, reflection, and translation of the plane.
\end{theorem}
\end{comment}