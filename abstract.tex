In political redistricting, the \textit{compactness} of a district is used as a quantitative proxy for its fairness.  Several well-established, yet competing, notions of geometric compactness are commonly used to evaluate the shapes of regions, including the Polsby-Popper score, the convex hull score, and the Reock score, and these scores are used to compare two or more districts or plans.  In this paper, we prove mathematically that, given any \textit{map projection} from the sphere to the plane, that there is some pair of regions whose score is reversed after the projection, for all three of these measures.  Finally, we demonstrate empirically the existence of legislative districts whose scores are permuted under the choice of map projection.  Our proofs use elementary techniques from geometry and analysis and should be accessible to an interested undergraduate.
