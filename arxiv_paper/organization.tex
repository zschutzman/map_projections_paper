\subsection{Organization}


For each of the compactness scores we analyze, our proof that no map
projection can preserve their order follows a similar recipe. We
first use the fact that any map projection which preserves an ordering
must preserve the \textit{maximizers} in that ordering.  In other words,
if there is some shape which a score says is \enquote{the most compact} on the sphere 
but the projection sends this to a shape in the plane which is \enquote{not the most compact}, then whatever 
shape \textit{does} get sent to the most compact shape in the plane leapfrogs 
the first shape in the induced ordering.  For all three of the scores we study, such a maximizer exists.

Using this observation, we can restrict our attention to those map
projections which preserve the maximizers in the induced ordering,
then argue that any projection in this restricted set must permute the
order of scores of some pair of regions.


\paragraph{Preliminaries}
We first introduce some definitions and results which we will use to prove our 
three main theorems.  Since spherical geometry differs from the more familiar planar geometry, 
we carefully describe a few properties of spherical lines and triangles to build some intuition 
in this domain. 


\paragraph{Convex Hull}
For the convex hull score, we first 
show that any projection which preserves the maximizers of the convex hull score 
ordering must maintain certain geometric properties of shapes and line segments 
between the sphere and the plane.  Using this, we demonstrate that no 
map projection from the sphere to the plane can preserve these properties, and therefore 
no such convex hull score order preserving projection exists.



\paragraph{Reock}
For the Reock score, we follow a similar tack, first showing that any 
order-preserving map projection must also preserve some geometric properties 
and then demonstrating that such a map projection cannot exist.





\paragraph{Polsby-Popper}
To demonstrate that there is no projection which maintains the score ordering induced by the Polsby-Popper score,  we leverage the 
difference between the \textit{isoperimetric inequalities} on the sphere and in the plane, in that the inequality for the plane is scale invariant in that setting but not on the sphere, in order to find a pair of regions in the sphere, one more compact than the other, such that the less compact one is sent to a circle under the map projection.

\paragraph{Empirical Results}
We finally examine the impact of the Mercator map projection on the convex hull, Reock, and Polsby-Popper 
scores and the ordering of regions under these scores.  While the impacts of the projection on the convex hull and 
 Polsby-Popper scores and their orderings are not severe, the Reock score and the Reock score ordering both change dramatically 
 under the map projection.




