\section{Organization}


For each of the compactness scores we analyze, our proof that no map
projection can preserve their order follows a similar recipe. We
first use the fact that any map projection which preserves an ordering
must preserve the \textit{maximizers} in that ordering.  In other words,
if there is some shape which a score says is `the most compact' on the sphere 
but this is not the case in the plane after applying a projection, then whatever 
shape \textit{does} get sent to `the most compact' shape in the plane leapfrogs 
the first shape in the induced ordering.

Using this observation, we can restrict our attention to those map
projections which preserve the maximizers in the induced ordering,
then argue that any projection in this restricted set must permute the
order of scores of some pair of regions.


\paragraph{Convex Hull}
For the convex hull score, we take a slightly different approach.  We first 
show that any projection which preserves the maximizers of the convex hull score 
ordering must maintain certain geometric properties of shapes and line segments 
between the sphere and the plane.  Using this, we use some spherical geometry demonstrate that no 
map projection from the sphere to the plane can preserve these properties, and therefore 
no such convex hull score order preserving projection exists.



\paragraph{Reock}
We  review some mathematical background related to 
the   stereographic projection, and showing 
that, in fact, any projection preserving maximizers 
 of the Reock score ordering must be stereographic. 
Then, we show that the  stereographic 
projection does not preserve the ordering induced Reock scores. 




\paragraph{Polsby-Popper}
To demonstrate that there is no projection which maintains the score ordering induced by the Polsby-Popper score, we again use properties of spherical geometry.  In this setting, we leverage the 
difference between the \textit{isoperimetric inequalities} on the sphere and in the plane in order to 
demonstrate that the Polsby-Popper score, as given, is not scale-free on the sphere.  We then readjust 
our definition to a more robust version of the isoperimetric quotient and use the machinery developed 
in the Reock section to demonstrate that no map projection exists which preserves the ordering induced by this score, either.

\paragraph{Discussion}
Finally, we discuss some generalizations of our work. While we prove our results in the contexts of three specific scores, the machinery we develop along the way can be broadly applied to other kinds of compactness scores.  

We defer to the appendix some generalizations of these results including the analogous results for several less commonly used compactness scores and some analogous results which apply to projections between surfaces of \textit{constant curvature}, rather than simply between the sphere and the plane.


