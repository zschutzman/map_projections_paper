In political redistricting, the \textit{compactness} of a district is
used as a quantitative proxy for its fairness.  Several
well-established, yet competing, notions of geographic compactness are
commonly used to evaluate the shapes of regions, including the
\textit{Polsby-Popper score}, the \textit{convex hull score}, and the \textit{Reock score}, and
these scores are used to compare two or more districts or plans.  In
this paper, we prove mathematically that  any \textit{map
projection} from the sphere to the plane reverses the ordering of the scores of some pair of regions for all three of these scores.  Empirically, we demonstrate that the effect of the common \textit{Mercator} map projection on the order of Reock scores is quite dramatic.
