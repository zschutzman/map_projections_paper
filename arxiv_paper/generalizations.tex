\section{Generalizations}

In the previous sections, we proved our non-preservation results for particular compactness scores which appear frequently in the context of electoral redistricting.  However, we remark here that the machinery and techniques can be applied to a wide range of plausible compactness measures.  We briefly present some of these other scores here and leave as exercises to the interested reader the proofs of non-preservation of score ordering.

\paragraph{Perimeter} 

The state of Iowa has in its laws \footnote{Iowa Code \S42.4(4)} the following as a measure of compactness:

\begin{displayquote}
	
	Perimeter compactness determines how regularly shaped a district in a
	redistricting plan is by measuring the distance to traverse the perimeter boundary of
	the district. The compactness of a district based on this measure is greatest when
	the distance needed to traverse the boundary is as short as possible. 
	
	\end{displayquote}

Prove that the compactness score defined by $\mathrm{Perim}(\Omega)$ does not have its ordering preserved by any map projection.



	
	