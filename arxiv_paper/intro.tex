\section{Introduction}
Striving for the \textit{geometric compactness} of legislative
districts is a traditional principle of redistricting \cite{altman_1998}, and, to that
end, many jurisdictions have included the criterion of compactness in
their legal code for drawing districts.  Some of these include Iowa's\footnote{Iowa Code \S42.4(4)} measuring the perimeter of districts, Maine's\footnote{Maine Statute \S1206-A} minimizing travel time within a district, and Idaho's\footnote{Idaho Statute 72-1506(4)} avoiding 
drawing districts which are `oddly shaped'.  Such measures can vary widely in their 
precision, both mathematical and otherwise.  Computing the perimeter of districts is a very clear definition, minimizing travel time is less so, and what makes a district oddly shaped or not seems rather challenging to consider from a rigorous standpoint. 


Mathematicians have studied geographic compactness for nearly two hundred years, and over that period, scientists and legal scholars have developed many formulas to compute the `compactness' of regions such as legislative districts \cite{young_compactness}.
 Three of the most commonly discussed formulations are the \textit{Polsby-Popper score}, which
measures the ratio of a district's area to the square of its
perimeter, the \textit{convex hull score}, which measures the ratio
of the area of a district to the smallest convex region containing it,
and the \textit{Reock score}, which measures the ratio of the area of
a district to the smallest circle containing it.  Each of these
measures is appealing at an intuitive level, since they assign to
a district a single scalar value between zero and one, which presents a simple 
method to compare the relative compactness of two or more districts. 
Additionally, the
mathematics underpinning each is widely understandable by the relevant
parties, including lawmakers, judges, advocacy groups, and the general
public.  

However, none of these measures truly discerns which districts are `compact' and which are not. 
For each score, we can construct a 
mathematical counterexample for which
a human's intuition and the score's evaluation of a shape's
compactness differ.  A region which is roughly circular but has a jagged boundary 
may appear compact to a human's eye, but such a shape has a very poor Polsby-Popper score.  Similarly, a very long, thin rectangle appears non-compact to a person, but has a perfect convex hull score.  Additionally, these scores often do not agree.
The long, thin rectangle has a  very
poor Polsby-Popper score, and the ragged circle has an excellent convex hull score.  These issues are well-studied by political
scientists and mathematicians alike
\cite{polsby1991third,frolov1975shape,maceachren1985compact}.




In this paper, we propose a further critique of these measures, namely
\textit{sensitivity under the choice of map projection}.  Each of the
compactness scores named above is defined as a tool to evaluate
geometric shapes in the plane, but in reality we are interested in
analyzing shapes which sit on the surface of the planet Earth, which
is (roughly) spherical.  
When we analyze the geometric properties of a geographic region, we work 
with a \textit{projection} of the Earth onto a flat plane, such as a piece of 
paper or the screen of a computer.
Therefore, when a shape is assigned a compactness score,
it is implicitly done with respect to some choice of map projection.
We prove that this may have
serious consequences for the evaluation of these scores.  In
particular, we consider the Polsby-Popper, convex hull,
and Reock scores on the sphere, and demonstrate that for any choice of
map projection, there are two regions, $A$ and $B$, such that $A$ is
more compact than $B$ on the sphere but $B$ is more compact than $A$
when projected to the plane.

\zs{i think the org section should be a subsection of the Intro}