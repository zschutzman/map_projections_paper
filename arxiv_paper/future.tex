\section{Discussion}



We have identified a major \textit{mathematical} weakness in the commonly discussed compactness scores in that no map projection can preserve the ordering over regions induced by these scores.  This leads to several important considerations in the mathematical and popular examinations of the detection of gerrymandering.

From the mathematical perspective, rigorous definitions of compactness require more nuance than the simple score functions which assign a single real-number value to each district.  \textit{Multiscale} methods, such as those proposed in in \cite{deford2018tv}, assign a vector of numbers or a function to a region, rather than a single number.  The richer information contained in such constructions is less susceptible to perturbations of map projections.
Alternatively, we can look to capturing the geometric information of a district without having to work with respect to a particular spherical or planar representation.  So-called \textit{discrete compactness} methods, such as those proposed in \cite{duchin2018discrete}, extract a graph structure from the geography and are therefore unaffected by the choice of map projection, and our results suggest that this is an important advantage of these kinds of scores over traditional ones.  Finally, recent work has used lab experiments to discern what qualities of a region humans use to determine whether they believe a region is compact or not \cite{kingeyeball}.  Incorporating more qualitative techniques is important, especially in this setting where the social impacts of a particular districting plan may be hard to quantify.


We proved our non-preservation results for three particular compactness scores which appear frequently in the context of electoral redistricting.  There are countless other scores offered in legal codes and academic writing, such as definitions analogous to the Reock and convex hull scores which use different kinds of bounding regions, scores which measure the ratio of the area of the largest inscribed shape of some kind to the area of the district, and versions of these scores which replace the notion of \enquote*{area} with the population of that landmass.  Many of these and others suffer from similar flaws as the three scores we examined in this work.  It would be interesting to consider the most general version of this problem and enumerate a collection of properties such that any map projection permutes the score ordering of a pair of regions under a score with at least one of those properties. 

While compactness scores are not used critically in a \textit{legal} context, they appear frequently in the popular discourse about redistricting issues and frame the perception of the \enquote*{fairness} of a plan.  An Internet search for a term like `most gerrymandered districts' will invariably return results naming-and-shaming the districts with the most convoluted shapes rather than highlighting where more pleasant looking shapes resulted in unfair electoral outcomes. 

Similarly, a sizable amount of work towards remedying such abuses focuses primarily on the geometry rather than the politics of the problem. Popular press pieces (e.g.\ \cite{ingraham2014solve}) and academic research alike (e.g.\ \cite{voronoi, svec2007applying, levin_friedler_2019}) describe algorithmic approaches to redistricting which use geometric methods to generate districts with appealing shapes.  
However, these approaches ignore all of the social and political information which are critical to the analysis of whether a districting plan treats some group of people unfairly in some way. 
Presenting purely geometric approaches to drawing districts implicitly presupposes that the mathematics used to evaluate the geometric features of districts are unbiased and unmanipulable and therefore provide true insight into the fairness of electoral districts.  We proved here that the use of geographic compactness as a proxy for fairness is much less clear and rigid than some might expect.




This work opens several promising avenues for further investigation.  We prove strong results for the most common compactness scores, but the question remains what the most general mathematical result in this domain might be, such as giving a set of necessary and sufficient conditions for a compactness score to not induce a permuted order for some choice of map projection.  