\section{Discussion and Future Directions}

\zs{this needs some work, but I think it's muy importante to include some discussion}

We have identified a major \textit{mathematical} weakness in the commonly discussed compactness scores in that no map projection can preserve the ordering over regions induced by these scores.  This leads to several important considerations in the mathematical and popular examinations of the detection of gerrymandering.

From the mathematical perspective, rigorous definitions of compactness require more nuance than the simple score functions which assign a single real-number value to each district.  \textit{Multiscale} methods assign a vector of numbers or a function to a region, rather than a single number.  The construction in \cite{deford2018tv} generalizes the Polsy-Popper and Ehrenberg scores and still captures geometric information about the region. It is still therefore susceptible to perturbations under the choice of map projection, but further study is warranted to determine to what degree this is a problem.
Alternatively, we can look to capturing the geometric information of a district without having to work with respect to a particular spherical or planar representation.  So-called \textit{discrete compactness} methods, such as those proposed in \cite{duchin2018discrete}, extract a graph structure from the geography and are therefore unaffected by the choice of map projection, and our results suggest that this is an important advantage of these kinds of scores over traditional ones.


While compactness scores are not used critically in a \textit{legal} context, they appear frequently in the popular discourse about redistricting issues and frame the perception of the `fairness' of a plan.  For example, a 2014 Washington Post piece  \cite{ingraham2014solve} describes an algorithm which generates highly compact districts because it ignores all of the social and demographic data which are crucial to the process.  The equating of `solving' gerrymandering with generating highly compact districts presupposes that the mathematics used to evaluate the geometric features of districts are unbiased and unmanipulable, and we demonstrate here that this is certainly not the case.


This work opens several promising avenues for further investigation.  We prove strong results for the most common compactness scores, but the question remains what the most general mathematical result in this domain might be, such as giving a set of necessary and sufficient conditions for a compactness score to not induce a permuted order for some choice of map projection.  