\section{Preliminaries, Definitions}\label{sec:prelims}

We begin by introducing the necessary definitions and terminology, as well as a few observations about the mathematical objects of interest which will be of use later.  We carefully lay out these definitions so as to align with an intuitive understanding of the concepts and to appease the astute reader who may be concerned with edge cases, geometric weirdness, and nonmeasurability.



\begin{definition}
A \textbf{region} $\Omega$ is an non-empty open set together with its boundary such that the region is measurable, its boundary is measurable, and it is connected.
\end{definition}

We choose this definition so that concepts of the `area' and `perimeter' of a region are well-defined concepts.  Throughout, we restrict our attention to the plane $\mathbb{R}^2$ (or $\C$ if one prefers) and the surface of the unit sphere $\mathbb{S}^2$ equipped with the standard measures and metrics.  We leave the consideration of other surfaces, measures, and metrics to future work.


\begin{definition}
A \textbf{compactness score function} $\mathcal{C}$ is a function from the set of all regions to the positive real numbers.  We adopt the convention that a region with a \textit{higher} compactness score is \textit{more} compact, and this naturally induces a partial order over the set of all regions, where $A$ is at least as compact as $B$ if and only if $\mathcal{C}(A)\geq \mathcal{C}(B)$.
\end{definition}

The final major definition we need is that of a \textit{map projection}.  In reality, the regions we are interested in comparing sit on the surface of the Earth (i.e. a sphere), but these regions are often examined as being projected onto a flat sheet of paper or computer screen, and which means that the regions drawn in any flat map object are subject to such a projection.

\begin{definition}
A \textbf{map projection} $\varphi$ is a local diffeomorphism from the sphere to the plane.  This means that $\varphi$ is continuous, $\varphi^{-1}$ exists and is also continuous, and the image of a region in the sphere is a region in the plane.
\end{definition}

\begin{definition}
We use the word \textbf{transformation} [of the plane/sphere] to mean to a diffeomorphism from the plane or sphere to itself.  \textbf{Linear transformations} of the plane are those parametrized by invertible $2{\times}2$ matrices, and \textbf{affine linear transformations} are those transformations which are the composition of a linear transformation and a translation of the plane.
\end{definition}




Since the image of a region under a map projection $\varphi$ is also a region, we can examine the compactness score of that region both before and after applying $\varphi$, and this is the heart of the problem we address in this paper.  We demonstrate, for several standard choices of compactness scores $\mathcal{C}$, that the order induced by $\mathcal{C}$ is different than the order induced by $\mathcal{C}\circ\varphi$ for \textit{any} choice of map projection $\varphi$.




For each of the compactness scores we analyze, our proof that no map projection can preserve their order follows a similar recipe.  We first use the fact that any map projection which preserves an ordering must preserve the \textit{maximizers} of that ordering, meaning that if $\Omega$ is a region for which $\mathcal{C}(\Omega)\geq \mathcal{C}(\Sigma)$ for all regions $\Sigma$, then it must at the very least be the case that $\mathcal{C}(\varphi(\Omega))\geq \mathcal{C}(\varphi(\Sigma))$ if $\varphi$ preserves $\mathcal{C}$'s ordering.

Using this fact, we can restrict our attention to those map projections which preserve the maximizers in the induced ordering, then argue that any projection in this restricted set must permute the order of scores of some pair of regions.