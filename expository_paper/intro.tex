\section{Introduction}\label{sec:intro}


Striving for the \textit{geometric compactness} of legislative districts is a traditional principle of redistricting, and, to that end, many jurisdictions have included the criterion of compactness in their legal code for drawing districts.  However, there is no agreed-upon definition for what makes a district compact or not.  Several competing mathematical definitions have emerged over the past two centuries, including the \textit{Polsby-Popper score}, which measures the ratio of a district's area to the square of its perimeter,  the \textit{convex hull score}, which measures the ratio of the area of a district to the smallest convex region containing it, and the \textit{Reock score}, which measures the ratio of the area of a district to the smallest circle containing it.  Each of these measures is appealing at an intuitive level, since they each assign to a district a single scalar value between zero and one, allowing easy comparisons between proposed redistricting plans. Additionally, the mathematics underpinning each is widely understandable by the relevant parties, including lawmakers, judges, advocacy groups, and the general public.  None of these measures is perfect, however.  For each, it is not difficult to construct a mathematical counterexample for which a human's intuition and the score's evaluation of a shape's compactness differ, such as a circle with slightly perturbed boundary for the Polsby-Popper measure and a very long, thin rectangle for the convex hull measure.  Additionally, these scores often do not agree.  The long, thin rectangle has a very good convex hull score, but a very poor Polsby-Popper score.  These issues are well-studied by political scientists and mathematicians alike \cite{polsby1991third,frolov1975shape,maceachren1985compact}.

In this paper, we propose a further critique of these measures, namely \textit{sensitivity under the choice of map projection}.  Each of the compactness scores named above is defined as a tool to evaluate geometric shapes in the plane, but in reality we are interested in analyzing shapes which sit on the surface of the planet Earth, which is (roughly) spherical.  When a shape is assigned a compactness score, it is implicitly done with respect to some choice of map projection.  We show, both mathematically and empirically, that this may have serious consequences for the evaluation of compactness.  In particular, we define the analogue of the Polsby-Popper, convex hull, and Reock scores on the sphere, and demonstrate that for any choice of map projection, there are two regions, $A$ and $B$, such that $A$ is more compact than $B$ on the sphere but $B$ is more compact than $A$ when projected to the plane.