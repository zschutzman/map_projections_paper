\section{Convex Hull}\label{sec:ch}
Another commonly used compactness score is the \textit{convex hull
score}.  We briefly recall the definition of a convex set and then
define this score function.

\begin{definition}
  Let $\mathrm{conv}(\Omega)$ denote the \textit{convex hull} of
  a region $\Omega$ in either the sphere or the plane, which is the
  smallest convex region containing $\Omega$.  Then we define the
  \textit{convex hull score} of $\Omega$ as 
  \begin{align*}
    \mathrm{CH}(\Omega)=
    \frac{\mathrm{area}(\Omega)}{\mathrm{area}(\mathrm{conv}(\Omega))}
  \end{align*}
\end{definition}

The convex hull score of a region $\Omega$  will be equal to one if
and only if $\Omega$ is itself convex.  We note that the convex hull
and Polsby-Popper scores agree when $\Omega$ is a circle, provide
a similar score when $\Omega$ is roughly circluar, and differ greatly
if $\Omega$ is a highly non-circular but convex region, such as
a long, thin rectangle.

Our strategy to demonstrate that any map projection cannot preserve
the orders of convex hull scores will be similar to the previous
section. We first argue that any order-preserving map must, in
particular, preserve the maximizers in the ordering, meaning convex
regions on the sphere are mapped to convex regions in the plane.
Then, we use this condition to restrict our attention to the map
projections which preserve these maximizers, which we can then use to
construct our counterexample.


We now show that the gnomonic projection does not preserve 
convex-hull score orderings on any region of the sphere.

\abn{mute this and write it in later:
This lemma lets us prove our theorem in two steps.  We can first show
that the gnomonic projection does not preserve the ordering of convex
hull scores, then argue that this misordering cannot be corrected by
any affine linear transformation of the plane.  This first part we
will prove by explicitly constructing two regions whose convex hull
scores are permuted under the gnomonic projection, and to facilitate
this construction, we make the following observation:}

\abn{Condensed all of this to one paragraph. See below.}
\mute{
\begin{example}
  Let $\varphi$ be the gnomonic projection and $C_\theta$ be a spherical
  cap centered at the south pole parametrized by the angle $\theta$
  formed between the central axis of the sphere and a line segment
  between the center of the sphere and the boundary of the cap. 
  Then $\varphi(C_\theta)$ is a disk in the
  plane, centered at the origin, and has radius $\tan(\theta)$.
  \begin{figure}
    \centering
    \includegraphics[width=.8\textwidth]{figs/gnom_cap.jpg}\\
    \caption{The image of a cap with polar angle $\theta$ is a circle of radius $\tan\theta$.}
    \label{fig:gnomcap}
  \end{figure}
  \begin{proof}
    A (literal) sketch of this proof can be seen in Figure~\ref{fig:gnomcap}.

    Since $\varphi$ projects from the center of the sphere and the
    sphere's south pole is mapped to the origin in the plane, the
    image of $C_\theta$ is totally radially symmetric about the
    origin, and is therefore a circle.  To see that its radius is
    $\tan(\theta)$, place the south pole of the sphere tangent to the
    plane at the origin. By construction, for any point $p$ on the
    boundary of $C_\theta$, there is a unique line passing through the
    center of the sphere, $p$, and the point $\varphi(p)$ on the
    boundary of the disk in the plane.  By definition, this line meets
    the central axis of the sphere at an angle of $\theta$, and the
    central axis of the sphere meets the plane orthogonally, so the
    center of the sphere, the origin, and the point $\varphi(p)$ form
    a right triangle with angle $\theta$.  Since we know that the
    distance between the center of the sphere and the origin is 1, we
    can write the distance between the origin and $\varphi(p)$ as
    $\tan(\theta)$.
  \end{proof}
  Using this, we can perform the construction of two regions whose
  convex hull scores are permuted by the gnomonic projection.

  \begin{lemma}
    There exist two regions on the sphere, $A$ and $B$, such that
    $\mathrm{CH}(A) > \mathrm{CH}(B)$ in the sphere, but, under the
    gnomonic projection $\varphi$,
    $\mathrm{CH}(\varphi(A))<\mathrm{CH}(\varphi(B))$.
  \end{lemma}

  Let $A$ be the region on the sphere defined by taking a cap centered
  at the south pole parametrized by the angle $\alpha_2$ and removing
  the cap centered at the south pole parametrized by the angle
  $\alpha_1$, with $\alpha_1<\alpha_2$.  Similarly, let $B$ be the
  region defined by angles $\beta_1$ and $\beta_2$.  The convex hull
  score of this kind of region on the sphere is $1-\frac{\text{area of
  inner cap}}{\text{area of outer cap}}$.  The projection under
  $\varphi$ of this kind of region is a disk in the plane with
  a smaller, cocentric disk deleted.  The convex hull score of this kind
  of region in the plane is $1-\frac{\text{area of inner
  disk}}{\text{area of outer disk}}$.  

  In the previous section, we parametrized a cap on the sphere by its
  \textit{height}, but we can instead parametrize it by its
  \textit{polar angle}, which is the angle formed between the polar axis
  and a line segment connecting the center of the sphere with the
  boundary of the cap.  Using the notation of $h$ for the height of
  a cap and $r$ for its radius as before, and letting $\theta$ be the
  cap's polar angle, we can use trigonometry to rewrite the area of the
  cap at height $h$, $\kappa(h)$.  Since $(1-h)=\cos(\theta)$, the area
  of the cap at height $h$ is $2\pi (1-\cos(\theta))$.  Therefore, if we
  have two cocentric caps on the disk parametrized by angles
  $\theta_1<\theta_2$, the convex hull score of this region is 

  $$1-\frac{2\pi (1-\cos(\theta_1))}{2\pi (1-\cos(\theta_2))}
  = 1-\frac{1-\cos(\theta_1)}{1-\cos(\theta_2)}.$$

  In the plane, since the image of a cap centered at the south pole
  parametrized by angle $\theta$ is a circle of radius $\tan^2(\theta)$,
  and the convex hull score of the image of a pair of cocentric caps
  parametrized by angles $\theta_1<\theta_2$ is

  $$  1-\frac{\tan^2(\theta_1)}{\tan^2(\theta_2)}.  $$

  Next, we observe that as we take a cap parametrized by an angle close
  to $\pi/2$ and increase that angle, the area of the cap grows much
  more slowly than the area of the disk defined by the cap's projection
  under $\varphi$.  This can be shown formally using
  calculus\footnote{The derivative of $1-\cos(\theta)$ is $\sin(\theta)$
  and the derivative of $\tan^2(\theta)$ is
  $2\tan(\theta)\sec^2(\theta)$.  The quantity
  $2\tan(\theta)\sec^2(\theta)-\sin(\theta)$ is positive for
  $0<\theta<\pi/2$.  We can also observe that as $\theta$ approaches
  $\pi/2$, $\tan^2(\theta)$ grows without bound, but $1-\cos(\theta)$
  approaches 1.} and intuitively recognized by considering that near
  $\theta=0$, $1-\cos(\theta)$ and $\tan^2(\theta)$ are almost equal,
  but as $\theta$ grows and approaches 1, $\tan^2(\theta)$ is much
  larger than $1-\cos(\theta)$.  What this means in the context of the
  gnomonic projection is that the areas of regions near the south pole
  of the sphere are not distorted much, but the areas of regions far
  from the south pole become very large under the projection, and
  increasingly so the further from the south pole they are. 

  We will use this observation to construct our example of two regions
  whose convex hull scores are permuted under $\varphi$, since for all
  such regions constructed as above for a fixed convex hull score on the
  sphere, those parameterized by larger angles will have comparably
  worse scores in the plane.  We will take two regions whose spherical
  convex hull scores are near $.4$, but perturbed slightly such that the
  one parametrized by larger angles has a slightly lower score.  The
  distortion of areas under the projection will more than compensate for
  this slight difference.


  Let $A$ be defined by the angles (in degrees) $\alpha_1=46^\circ$ and
  $\alpha_2=60^\circ$ and $B$ be defined by $\beta_1=20^\circ$ and
  $\beta_2=26^\circ$.  Then we have $\mathrm{CH}(A)\approx .39$ and
  $\mathrm{CH}(B)\approx .41$, but $\mathrm{CH}(\varphi(A))\approx .64$
  and $\mathrm{CH}(\varphi(B))\approx .45$, which is our example of two
  regions whose scores' order are permuted by $\varphi$.
\end{example}}


\begin{example}
  Let $p = (0,0,1)$ be the center of the sphere, and let $0$ be the
  origin. For any angle $0\le \theta< \frac\pi2$, let $C_\theta$ be
  the spherical cap:
  \begin{align*}
    C_{\theta} = \{x|0\le\angle(x,o,p)\le \theta\}
  \end{align*}
  Let $A(\alpha_1,\alpha_2) = C_{\alpha_2}\ssm C_{\alpha_1}$ be 
  an annular region. A straightforward computation shows that:
  \begin{align*}
    \mathrm{CH}(A(\alpha_1,\alpha_2)) &= 
    1-\frac{\mathrm{Area}(C_{\alpha_1})}{\mathrm{Area}(C_{\alpha_2})}
    =1- \frac{1-\cos(\alpha_1)}{1-\cos(\alpha_2)}\\
    \mathrm{CH}(\vphi(A(\alpha_1,\alpha_2))) &=
    1-\frac{\mathrm{Area}(\vphi(C_{\alpha_1}))}%
    {\mathrm{Area}(\vphi(C_{\alpha_2}))}
      =1- \frac{\tan^2(\alpha_1)}{\tan^2(\alpha_2)}
  \end{align*}
  Where $\vphi$ is the gnomonic projection. 
  Let $A=A(\alpha_1,\alpha_2)$, and $B = A(\beta_1,\beta_2)$. 
  Let $\alpha_2$ be arbitrary, and choose 
  $\alpha_1 = \frac{\alpha_2}{2} = \beta_2$, and 
  $\beta_1 = \frac{\beta_2}{2}$. 
\end{example}
Note that in the above example, $\alpha_1$ (and hence 
all other angles) can be chosen to be arbitrarily small. 

Now that we've shown that the gnomonic projection alone cannot
preserve convex hull scores, to complete the argument, we must now
show that there is no affine linear transformation of the plane which
can correct for this.

\begin{lemma}\label{lem:noafflin}
  Let $L$ be an affine linear transformation of the plane.  Then $L$
  preserves the convex hull scores of all figures in the plane.
\end{lemma}
\begin{proof}
  Since affine linear transformations map lines to lines, they preserve
  convexity, and since they also preserve the containment of regions in
  other regions, if $\Omega$ is a region in the plane, then it must be
  the case that the convex hull of $\Omega$ is mapped to the convex hull
  of $L(\Omega)$ by $L$.  In other words, the image of the convex hull
  of $\Omega$ is the convex hull of the image of $\Omega$.  Finally,
  since affine linear transformations preserve the ratio of the areas of
  any two regions, $L$ must, in particular, preserve the ratio of the
  areas of $\Omega$ and its convex hull, so the convex hull score is
  preserved.
\end{proof}

We finally have the tools to prove the main result of this section,
which is a direct consequence of the four previous lemmas.

\begin{theorem}
  There is no map projection from the half-sphere to the plane which
  preserves the ordering of convex hull scores for all regions.
\end{theorem}

\begin{proof}
  Since such a projection must map convex regions on the sphere to
  convex regions in the plane, it must be a composition of the gnomonic
  projection followed by an affine linear transformation of the plane.
  Since affine linear transformations of the plane preserve convex hull
  scores and therefore preserve their orders, a convex hull score
  order-preserving projection from the sphere to the plane cannot exist
  if the gnomonic projection does not preserve their orders.  By the
  counterexample we constructed, it does not, and therefore there is no
  projection from the sphere to the plane which preserves the ordering
  of convex hull scores for all regions.
\end{proof}
